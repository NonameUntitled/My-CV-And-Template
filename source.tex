% basic part begin
\documentclass[10pt,a4paper]{report}
\usepackage[utf8]{inputenc}
\usepackage[russian]{babel}
\usepackage[OT1]{fontenc}
\usepackage{amsmath}
\usepackage{amsfonts}
\usepackage{amssymb}
\usepackage[misc]{ifsym}
\usepackage{makeidx}
\usepackage{wasysym}
\usepackage{enumitem}
\usepackage{csquotes}
\usepackage{microtype}
\usepackage{vwcol}
\usepackage{tgbonum}
\usepackage[dvipsnames]{xcolor}


% basic part end

\usepackage{multicol}
\usepackage{fontawesome}
\usepackage{hyperref}
\usepackage[left=1cm,right=1cm,top=1cm,bottom=1cm]{geometry}

\setromanfont{LeagueSpartan}[
    Path=./Font/,
    Extension = .ttf,
    UprightFont=*-Regular,
    BoldFont=*-SemiBold
    ]

\author{Vladimir Baikalov}

\setlength\columnsep{15mm}
\setlength\parindent{0pt}
\pagenumbering{gobble}

\begin{document}

\noindent
\begin{minipage}[c]{0.66\linewidth}
\Huge{\textbf{Vladimir Baikalov}}

\small{Machine Learning Engineer/Deep Learning Engineer}
\end{minipage} % no space if you would like to put them side by side
\begin{minipage}[c]{0.33\linewidth}
\Letter: nonameuntitled159@gmail.com

\faLinkedin : \href{https://www.linkedin.com/in/noname-untitled}{linkedin.com/in/noname-untitled}

\faGithub:  \href{http://www.github.com/NonameUntitled}{github.com/NonameUntitled}

\faSend:  \href{http://www.t.me/noname\_untitled}{t.me/noname\_untitled}

\end{minipage}

\par\hbox{\Large{\textcolor{Plum}{\textbf{Work experience}}}}\kern3pt\hrule\kern5pt

\textbf{\large{Machine Learning Engineer/}}\textcolor{Plum}{Google, YouTube.}
\hfill
\textbf {\underline{January 2023 - Present}}

Paris, France

\textbf{Relevant areas}: Transformers, Highload, Large Language Models (LLMs)

\begin{itemize}[label={}, left=10pt]
    \item  Developed algorithm for video games detection for shorts. The final model achieved \textbf{4\% recall boost without precision drop} and allowed to \textbf{conduct fine-tune more frequently}.

    \item Currently working on \textbf{deployment of LLMs} in the production including improvement of \textbf{re-training and distillation} pipelines. 

\end{itemize}

\vspace{2mm}

\textbf{\large{Deep Learning Engineer/}}\textcolor{Plum}{Yandex Technology.}
\hfill
\textbf {\underline{August 2021 - December 2022}}

Moscow, Russia

\textbf{Relevant areas}: Transformers, Highload, Recommended Systems
\begin{itemize}[label={}, left=10pt]

    \item  Transferred ML model from experimental setup (Python, Pytorch) to production framework (C++, YNMT). Supported weekly continuous finetuning process for in \textbf{Yandex.Ads}. \textbf{It was applied in production}. This model increased \textbf{GMV up to 1.5\%} and a number \textbf{of clicks up to 5\%}. The result is verified by AB tests and experiments.
    
    \item Applied encoder-based models for improving personalized ads and search recommendations. Current solution \textbf{boosts production metric up to 6\%}.
    
    \item Implement multiprocessing Python package (YtReader) for fast and convenient data preprocessing. The final solution reduced the time required for models training/evaluating \textbf{up to 5 times}. 
\end{itemize}

\textbf{\large{Machine Learning Engineer}}/\textcolor{Plum}{Huawei R\&D Dept.}
\hfill
\textbf {\underline{March 2020 - July 2021}}

Saint-Petersburg, Russia

\textbf{Relevant areas}: Object Detection/Tracking, Digital Sound Processing, Optical Character Recognition.

\begin{itemize}[label={}, left=10pt]
    \item Algorithm for vehicles trajectories prediction using radar data only. This approach now is being used in a real-world application. Proposed solution \textbf{over-performs the previous algorithm on 44\%}. 
    
    \item Car occupation algorithm for outdoor and underground parking lots based on detection and tracking algorithms. This solution was presented to the product team for further implementation in the product.
    
    \item Algorithm for the knuckle-knock sound pattern detection. Proposed architecture achieved \textbf{90\%} Precision and~\textbf{95\%} Recall. With touch sensors usage Precision was improved up to \textbf{94\%}.
\end{itemize}


\textbf{\large{Researcher/}}\textcolor{Plum}{ITMO University, ML Lab}
\hfill
\textbf {\underline{September 2020 - July 2021}}

Saint-Petersburg, Russia

\textbf{Relevant areas}: Reinforcement Learning

\begin{itemize}[label={}, left=10pt]
    \item Developed multi-agent policy-based algorithm \textbf{REM (Reinforce, Embedding, Monte-Carlo)} for baggage handling system (BHS), which over-performs previous approaches. This result was statistically proven with received \textbf{p-value less than~0.01}.
    
    \item The paper, \textbf{\href{https://aaltodoc.aalto.fi/handle/123456789/111642}{\enquote{Multi-Agent Deep Reinforcement Learning-Based Algorithm For Fast Generalization On Routing Problems}}},
    with the partial description of this project, was \textbf{published on the YSC 2021 conference}. This project was held in \textbf{collaboration with Aalto University, Finland}.
    
\end{itemize}

\par\hbox{
\Large{\textcolor{Plum}{\textbf{Education}}}
}{\kern5pt\hrule\kern5pt}

\textbf{\large{Master degree/}}\textcolor{Plum}{Skoltech, DS major}
\hfill
\textbf{\underline{September 2021 - Present}}

Moscow, Russia \hfill Current GPA: 5.00/5.00

\hspace{6mm}
Thesis project: Attention mechanism acceleration via tensor and matrix decompositions

\vspace{3mm}

\textbf{\large{Bachelor degree}}\textcolor{Plum}{/ITMO University, CS major}
\hfill
\textbf{\underline{September 2017 - August 2021}}

Saint-Petersburg, Russia \hfill GPA: 4.65/5.00

\hspace{6mm}
Thesis project: Implementing multi-agent policy-based algorithm for conveyor systems

\vspace{3mm}
\par\hbox{\Large{\textcolor{Plum}{\textbf{Skills}}}}\kern5pt\hrule
\begin{multicols}{3}

\textbf{\large{Programming languages}}

Proficient: Python, C++

Advanced: Java, Bash, SQL

\textbf{\large{Technologies}}

Proficient: Pytorch, Tensorflow

Advanced: Airflow, PySpark

\textbf{\large{Extra}}

English: IELTS C1 Certified

ICPC Contest Volunteering

\end{multicols}

\end{document}
